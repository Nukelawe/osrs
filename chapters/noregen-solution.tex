\subsection{Solving the recurrences}
We begin by defining $r \equiv \frac{m+1}{m}$ and $p \equiv \frac{-1}{m+1}$. This allows rewriting the recurrence relations more compactly as
\begin{align}\label{eq:complexRecurrence2}
	\boxed{\langle L_h \rangle = \begin{cases}
        r\left(\langle L_{h-1}\rangle - p\langle L_{h-m-1}\rangle \right) &\quad \mbox{if } h \leq m+1\\
        r\langle L_{h-1}\rangle &\quad \mbox{if } h > m+1
	\end{cases}}
\end{align}
with the initial condition $\langle L_1 \rangle = \frac{r}{a}$.
The case $h \leq m+1$ is simply the recurrence relation for a geometric sequence. Therefore,
\begin{align}\label{eq:geomProgression}
    \langle L_h \rangle &= \frac{r^h}{a} \quad \mbox{if } h \leq m+1.
\end{align}

The recurrence relation for the case $h > m+1$ is not as easy to bring in a non-recursive form. It is a linear recurrence of order $m$, the initial $m$ elements of which are given by eq\ref{eq:geomProgression}.
This recurrence is perhaps most effectively approached with generating functions. The ordinary generating function of the number sequence $\langle L_h \rangle$ is the function $g(z)$ whose power series expansion has $\langle L_h \rangle$ as coefficients.
\begin{align}
    g(z) = \sum_{h=1}^{\infty} \langle L_h \rangle z^h
\end{align}
Using the generating function formulation the recurrence (eq\ref{eq:complexRecurrence2}) translates into an algebraic equation for $g(z)$.
\begin{align}
    g(z) &= \sum_{h=1}^{m+1} \langle L_h \rangle z^h + \sum_{h=m+2}^{\infty} \langle L_h \rangle z^h\nonumber\\
         &= \sum_{h=1}^{m+1} \langle L_h \rangle z^h + \sum_{h=m+2}^{\infty} r\left(\langle L_{h-1}\rangle + p\langle L_{h-m-1}\rangle\right)z^h\nonumber\\
         &= \sum_{h=1}^{m+1} \langle L_h \rangle z^h + rz\sum_{h=m+2}^{\infty}\langle L_{h-1}\rangle z^{h-1} + rpz^{m+1}\sum_{h=m+2}^{\infty}\langle L_{h-m-1}\rangle z^{h-m-1}\nonumber\\
         &= \sum_{h=1}^{m+1} \langle L_h \rangle z^h + rz\sum_{h=m+1}^{\infty}\langle L_h\rangle z^{h} + rpz^{m+1}\sum_{h=1}^{\infty}\langle L_{h} \rangle z^{h}\nonumber\\
         &= \sum_{h=1}^{m} \langle L_h \rangle z^h + \langle L_{m+1} \rangle z^{m+1} + rz\left(\sum_{h=1}^{\infty}\langle L_{h} \rangle z^{h} - \sum_{h=1}^{m}\langle L_{h}\rangle z^{h} \right) + rpz^{m+1}g(z)\nonumber\\
         &= (1 - rz)\sum_{h=1}^{m} \langle L_h \rangle z^h + \langle L_{m+1}\rangle z^{m+1} + \left(rz + rpz^{m+1}\right)g(z)
\end{align}
Now, moving all terms containing $g(z)$ to the left and substituting the values for $\langle L_h \rangle$ from eq\ref{eq:geomProgression} gives
\begin{align}
    (1 - rz - rpz^{m+1})g(z) &= \frac{1}{a}\left({(rz)}^{m+1} + (1 - rz)\sum_{h=1}^{m} {(rz)}^h\right)\label{eq:ogfderiv1}
\end{align}
The partial sum on the right is the $m$ first terms of a geometric series which has the following closed form expression.
\begin{align*}
    \sum_{h=1}^{m} {(rz)}^h = rz\frac{1 - {(rz)}^m}{1-rz}
\end{align*}
Plugging it back to eq\ref{eq:ogfderiv1} to gives
\begin{align}
    (1 - rz - rpz^{m+1})g(z)
		&= \frac{1}{a}\left({(rz)}^{m+1} + rz\left(1 - {(rz)}^m\right)\right)\nonumber
        = \frac{1}{a}rz\nonumber\\
    \implies g(z) &= \frac{rz/a}{1 - rz - rpz^{m+1}}\label{eq:ogf}.
\end{align}

To read off the explicit formula for $\langle L_h \rangle$ the generating function must be expanded back into a power series. This is possible by noticing that $g(z)$ is an infinite geometric sum with $rz(1 - pz^m)$ as its coefficient.
\begin{align}
    g(z) &= \frac{rz/a}{1 - rz\left(1 + pz^{m}\right)}\nonumber
    = \frac{rz}{a} \sum_{n=0}^\infty {(rz(1 + pz^m))}^n\nonumber
    = \frac{1}{a} \sum_{n=0}^\infty {(rz)}^{n+1}{(1 + pz^m)}^n
\end{align}
Then, we expand the $n$th order binomial term using the binomial theorem.
\begin{align}
    g(z) &= \frac{1}{a} \sum_{n=0}^\infty {(rz)}^{n+1}\sum_{i=0}^{n} {(pz^m)}^i {n \choose i}\nonumber\\
         &= \sum_{n=0}^\infty \frac{1}{a}\sum_{i=0}^{n} r^{n+1} p^i {n \choose i}z^{mi+n+1}\label{eq:ogf2}.
\end{align}
$\langle L_h \rangle$ is now the coefficient of the term $z^h$. Looking at the exponent of $z$ in eq\ref{eq:ogf2} we see that for the coefficient of interest the indices $i$ and $n$ must satisfy $h=mi+n+1$. For each $i$ there is exactly one valid $n$, namely $n = h-mi-1$.
Therefore, the coefficient of the $h$th order term in the series expansion of $g(z)$ is
\begin{align}
    \langle L_h \rangle &= \frac{1}{a}\sum_{i \in \mathcal{I}} r^{h-mi} p^i {h-mi-1 \choose i} \nonumber
\end{align}
where $\mathcal{I}$ is the set of indices $i$ such that
\begin{align*}
    0 \leq i \leq n = h-mi-1
    \iff (m+1)i \leq h-1
    \iff i \leq \frac{h-1}{m+1}.
\end{align*}
Since $i$ must also be an integer the index set becomes $\mathcal{I} = \left\{0,1,\ldots,\left\lfloor{\frac{h-1}{m+1}}\right\rfloor\right\}$. Expressing the constants $r$ and $p$ explicitly, the solution to the recursion problem (eq\ref{eq:complexRecurrence2}) can be written as
\begin{align}
    \langle L_h \rangle &= \frac{1}{a}\sum_{i=0}^{\left\lfloor{\frac{h-1}{m+1}}\right\rfloor} {\left(\frac{m+1}{m}\right)}^{h-mi} {\left(\frac{-1}{m+1}\right)}^i {h-mi-1 \choose i}.\label{eq:explicitL}
\end{align}
Naturally this can also be proven to satisfy the recursion using induction (see Appendix~\ref{sect:genLProof}). With this result the damage dealt per hit is simply $h/\langle L_h \rangle$ and the damage per second can be obtained from it by dividing with the attack interval $T_A$.
