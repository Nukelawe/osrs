Most activities in \href{https://oldschool.runescape.com}{Old School RuneScape}~\cite{osrs} are repetitive and feature randomness, making it important to optimize the related expected value quantities. In the case of combat, the subject of optimization is typically either the kill rate or the damage rate (DPS), and a method for calculating them is desirable. The main difficulty in these calculations is caused by overkill, which occurs when the enemy does not have enough hitpoints to receive all the damage that was rolled by the attaker. Extra complexity is added by natural regeneration, which affects almost every fight to some, albeit small extent.
Earlier attempts at calculating the kill and damage rates have all made the assumption of no natural regeneration. An approximation for the damage rate was given by~\href{https://imgur.com/aykEahg}{Nukelawe}~\cite{nukelawe} using simplifying assumptions about the long term enemy hitpoint distribution. A more accurate model was presented by \href{https://www.reddit.com/r/2007scape/comments/faz5et/the_mathematics_of_osrs_combat/}{Palfore}~\cite{palfore} and an attempt at an exact solution for the regenerationless case was made by \href{https://www.reddit.com/r/2007scape/comments/bcq3mj/overkill_dps_formulas}{Corpslayer}~\cite{corpslayer1} who also derived an~\href{https://imgur.com/a/6613Tlu}{asymptotic approximation}~\cite{corpslayer2}.

In this study we will drop the assumption of no regeneration and treat Oldschool Runescape fights as absorbing Markov chains. This will allow us to obtain an exact solution for the expected number of hits to kill and the expected number of hitpoints regenerated. From these quantities, we will be able to calculate the kill and damage rates as long time averages. In search of less computationally intensive models and a better understanding of the solution we will additionally study in detail the case of no regeneration as well as some other approximations.
