\section{Fights with no regeneration}\label{chap:noregen}
To compute the length of a fight from the recurrence (eq\ref{eq:fightLengthRecursion}) we need to know the transition probabilities. If regeneration is ignored all the state transitions are caused by hitting the enemy. From the fight mechanics as stated in Chapter~\ref{chap:fightMechanics} we can determine the probability distribution of the accuracy-corrected damage roll $X$. This is the amount of damage dealt before capping it by overkill.
\begin{align}
	\Pr{X = k} =
	\begin{cases}
		1 - \frac{am}{m+1}, &\mbox{if } k = 0 \\
		\frac{a}{m+1},      &\mbox{if }1 \leq k \leq m
	\end{cases}\label{eq:damageRollDistribution}
\end{align}
where $a$ is the accuracy. As expected, the distribution is uniform everywhere except at 0, where the chance of missing skews it. In terms of eq\ref{eq:damageRollDistribution} the transition probabilities are given by
\begin{align}
    p_{ij}
         &= \begin{cases}
            \Pr{X = j-i} \quad &\mbox{if } i > 0 \\
            \Pr{X \geq j-i} \quad &\mbox{if } i = 0
        \end{cases}\label{eq:damageDistribution}.
\end{align}

Now we insert the transition probabilities into eq\ref{eq:fightLengthRecursion}. While doing so the case $X=0$ has to be treated separately as its probability has a different definition. Since the enemy can never have more than $h$ remaining hitpoints it suffices to consider $\L_j$ for $j \leq h$.
\begin{align}
    \L_j
        &= 1 + \sum_{i=1}^h p_{ij}\L_i\nonumber\\
        &= 1 + \sum_{i=1}^j \Pr{X = j-i}\L_i \qquad &\mbox{since $\Pr{X<0} = 0$}\nonumber\\
        &= 1 + \Pr{X=0}\L_j + \sum_{i=1}^{j-1} \Pr{X = j-i}\L_i\nonumber\\
    \implies \left(1 - \Pr{X=0}\right)\L_j &= 1 + \sum_{i=1}^{j-1} \Pr{X = j-i}\L_i\label{eq:noregen1}
\end{align}
$h$ cancelling out implies that it makes no difference if the enemy being fought has full hitpoints or if some of them are lost. In the absence of regeneration a fight against an enemy with $j$ hitpoints remaining is identical to a fight against an enemy whose maximum hitpoints are $j$. Therefore we might as well restrict ourselves to the case $j=h$.

Let us compute the damage roll probabilities appearing in eq\ref{eq:noregen1} (after the substitution $j = h$)
\begin{align}
    &\Pr{X=h-i} = \frac{a}{m+1}, \quad\mbox{if } h-m \leq i \leq h-1\\
    &\Pr{X=0} = 1 - \frac{ma}{m+1}
\end{align}
and insert them back to eq\ref{eq:noregen1} to get
\begin{align}
    &\frac{am}{m+1}\L_h = 1 + \frac{a}{m+1}\sum_{\mathclap{i=\max(1,h-m)}}^{h-1}\L_i\nonumber\\
	\implies &m\L_h = \frac{m+1}{a} + \quad\sum_{\mathclap{i=\max(1,h-m)}}^{h-1}\L_i\label{eq:derivation1}.
\end{align}
The simplest case $h=1$ gives
\begin{align}
    \L_1 = \frac{m+1}{a}
\end{align}
In the case $1 < h \leq m+1$ eq\ref{eq:derivation1} reduces to
\begin{alignat}{3}
    &m\L_h           &\;=\;& \frac{m + 1}{a} + \sum_{i=1}^{h-2}\L_{i} + \L_{h-1}\nonumber\\
    &                &\;=\;& m\L_{h-1} + \L_{h-1}\nonumber\\
    \implies &\L_{h} &\;=\;& \frac{m+1}{m}\L_{h-1}\label{eq:geometricRecurrence}
\end{alignat}
and if $h > m+1$, then
\begin{align}
    m\L_h
          &= \frac{m + 1}{a}\ +\ \sum_{\mathclap{i=(h-1)-m}}^{\mathclap{(h-1)-1}} \L_i + \L_{h-1} - \L_{h-m-1}\nonumber\\
          &= m\L_{h-1} + \L_{h-1} -\L_{h-m-1}\nonumber\\
    \implies \L_{h} &= \frac{m+1}{m}\L_{h-1} -\frac{1}{m}\L_{h-m-1}\label{eq:complexRecurrence}
\end{align}
These recurrences will be solved in Chapter~\ref{chap:recurrenceAnalysis}.
