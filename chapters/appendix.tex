\appendix
\section{Inductive proof of equation~\ref{eq:explicitL}}\label{sect:genLProof}
We will use the binomial coefficient identities ${k \choose n} = {k-1 \choose n} + {k-1 \choose n-1}$ and ${k \choose 0} = 1$ to prove that
\begin{align}
    \L_h &= \frac{1}{a}\sum_{i=0}^{N} r^{h-im}p^{i}{h-im-1 \choose i}, \quad \mbox{where } N = \left\lfloor{\frac{h-1}{m+1}}\right\rfloor\nonumber
\end{align}
is a solution to the recursion problem stated by eq\ref{eq:complexRecurrence2}.
\begin{proof}
If $0 < h \leq m+1$, then $N = 0$ and
\begin{align}
	\L_h &= \frac{1}{a}\sum_{i=0}^{0} r^{h-im}p^{i}{h-im-1 \choose i}\nonumber
    = a^{-1}r^{h}{h-1 \choose 0}\nonumber
         = a^{-1}r^{h}\nonumber
\end{align}
Therefore the statement holds for the initial condition.\\
Assume that the statement is true for the $m+1$ consecutive elements $\L_{h-1-m},\ldots,\L_{h-1}$. Then
\begin{align}
	\L_h &= r\left(\overline{L}_{h-1} + p\overline{L}_{h-m-1}\right)\nonumber\\
         &= r\left(\frac{1}{a}\sum_{i=0}^{\left\lfloor{\frac{(h-1)-1}{m+1}}\right\rfloor} r^{(h-1)-im}p^{i}{(h-1)-im-1 \choose i} +
         p\frac{1}{a}\sum_{i=0}^{\left\lfloor{\frac{(h-1-m)-1}{m+1}}\right\rfloor} r^{(h-1-m)-im}p^{i}{(h-1-m)-im-1 \choose i}\right)\nonumber\\
   \implies a\L_h &= r\sum_{i=0}^{\left\lfloor{\frac{h-2}{m+1}}\right\rfloor} r^{h-1-im}p^{i}{h-2-im \choose i} +
         rp\sum_{i=0}^{\left\lfloor{\frac{h-1-(m+1)}{m+1}}\right\rfloor} r^{h-1-(i+1)m}p^{i}{h-2-(i+1)m \choose i}\nonumber\\
         &= \sum_{i=0}^{\left\lfloor{\frac{h-2}{m+1}}\right\rfloor} r^{h-im}p^{i}{h-2-im \choose i} +
         \sum_{i=0}^{\left\lfloor{\frac{h-1-(m+1)}{m+1}}\right\rfloor} r^{h-(i+1)m}p^{i+1}{h-2-(i+1)m \choose i}\nonumber\\
         &= \sum_{i=0}^{\left\lfloor{\frac{h-2}{m+1}}\right\rfloor} r^{h-im}p^{i}{h-2-im \choose i} +
         \sum_{i=0}^{\left\lfloor{\frac{h-1}{m+1}}\right\rfloor-1} r^{h-(i+1)m}p^{i+1}{h-2-(i+1)m \choose i}\label{eq:proofBranch}
\end{align}
If $h-1 = N(m+1)$, then $\left\lfloor{\frac{h-2}{m+1}}\right\rfloor = \left\lfloor{\frac{h-1}{m+1}}\right\rfloor-1 = N-1$. Now eq\ref{eq:proofBranch} reduces to
\begin{align*}
	a\L_h &= \sum_{i=0}^{N-1} r^{h-im}p^{i}{h-2-im \choose i} +
         \sum_{i=0}^{N-1} r^{h-(i+1)m}p^{i+1}{h-2-(i+1)m \choose i}\\
         &= \sum_{i=0}^{N-1} r^{h-im}p^{i}{h-2-im \choose i} +
         \sum_{i=1}^{N} r^{h-im}p^{i}{h-2-im \choose i-1} \\
         &= \sum_{i=1}^{N} r^{h-im}p^{i}\left({h-2-im \choose i} + {h-2-im \choose i-1}\right)
         + r^{h}{h-2 \choose 0} - r^{h-Nm}p^{N}{h-2-Nm \choose N}\\
         &= \sum_{i=1}^{N} r^{h-im}p^{i}{h-1-im \choose i} + r^h -
         r^{N(m+1)+1-Nm}p^{N}{N(m+1)-1-Nm \choose N}\\
         &= \sum_{i=1}^{N} r^{h-im}p^{i}{h-1-im \choose i} + r^{h-0m}p^0{h-1-0m \choose 0} -
         r^{N+1}p^{N}\cancelto{0}{{N-1 \choose N}}\\
         &= \sum_{i=0}^{N} r^{h-im}p^{i}{h-1-im \choose i}
\end{align*}
which is the required form.

\noindent The other case to consider is $h-1 \neq N(m+1) \implies \left\lfloor{\frac{h-2}{m+1}}\right\rfloor = \left\lfloor{\frac{h-1}{m+1}}\right\rfloor = N$. In this case eq\ref{eq:proofBranch} gives
\begin{align*}
    a\L_h &= \sum_{i=0}^{N} r^{h-im}p^{i}{h-2-im \choose i} +
          \sum_{i=0}^{N-1} r^{h-(i+1)m}p^{i+1}{h-2-(i+1)m \choose i}\\
          &= \sum_{i=0}^{N} r^{h-im}p^{i}{h-2-im \choose i} +
          \sum_{i=1}^{N} r^{h-im}p^{i}{h-2-im \choose i-1}\\
          &= \sum_{i=1}^{N} r^{h-im}p^{i}\left({h-2-im \choose i} +
          {h-2-im \choose i-1}\right) + r^{h-0m}p^{0}{h-2-0m \choose 0}\\
          &= \sum_{i=1}^{N} r^{h-im}p^{i}{h-1-im \choose i} + r^{h-0m}p^{0}{h-1-0m \choose 0}\\
          &= \sum_{i=0}^{N} r^{h-im}p^{i}{h-1-im \choose i}
\end{align*}
This covers all the cases. If the statement is true for $h \in \{k-m-1,\ldots,k-1\}$, it will also be true for $h = k$. Since we have shown that the statement holds for the initial condition ($0 < h \leq m+1$) it follows by induction that it holds for all $h > 0$.
\end{proof}

\section{Derivation of the regenerative recurrence relations}\label{sect:regenRecurrenceDerivation}
First, consider fights that do not start at full hitpoints, i.e. $j < h$. Then
\begin{align}
	\langle L_{j-1} \rangle
		&= 1 + \sum_{i=1}^{j} p_{i,j-1}\langle L_i\rangle\nonumber\\
		&= 1 + \sum_{i=1}^{j-1} p_{i,j-1}\langle L_i\rangle + p_{j,j-1}\langle L_{j}\rangle\nonumber\\
	\implies p_{j,j-1}\langle L_{j}\rangle
		&= \langle L_{j-1} \rangle - 1 - \sum_{i=1}^{j-1} p_{i,j-1}\langle L_i\rangle\nonumber\\
		&= (1 - p_{j-1,j-1})\langle L_{j-1} \rangle - 1 - \sum_{i=1}^{j-2} p_{i,j-1}\langle L_i\rangle\nonumber\\
		&= 1 + \sum_{i=1}^{j} p_{ij}\langle L_i\rangle + p_{j+1,j}\left(1 + \sum_{i=1}^{j+2} p_{i,j+1}\langle L_i\rangle\right)\nonumber\\
		&= 1 + p_{j+1,j} + \sum_{i=1}^{j} p_{ij}\langle L_i\rangle + p_{j+1,j}\sum_{i=1}^{j+2} p_{i,j+1}\langle L_i\rangle\nonumber\\
		&= 1 + p_{j+1,j} + \sum_{i=1}^{j} \left(p_{ij} + p_{j+1,j}p_{i,j+1}\right)\langle L_i\rangle + p_{j+1,j}\sum_{i=j+1}^{j+2} p_{i,j+1}\langle L_i\rangle\nonumber\\
		&= 1 + p_{1j}\langle L_1 \rangle + \sum_{i=2}^{h-1} p_{ij}\langle L_i \rangle + p_{hj}\langle L_h\rangle\nonumber\\
		&= 1 + p_{1j}\langle L_1 \rangle + \sum_{i=2}^{h-1} p_{ij}\langle L_i \rangle + p_{hj}\langle L_h\rangle\nonumber\\
\end{align}
\begin{align}
    \langle L_h \rangle
		&= 1 + \sum_{i=1}^{h} p_{ih}\langle L_i\rangle\nonumber\\
		&= 1 + \sum_{i=1}^{j+1} p_{ij}\langle L_i\rangle\nonumber\\
		&= 1 + \sum_{i=1}^{j} p_{ij}\langle L_i\rangle + p_{j+1,j}\langle L_{j+1}\rangle\nonumber\\
		&= 1 + \sum_{i=1}^{j} p_{ij}\langle L_i\rangle + p_{j+1,j}\left(1 + \sum_{i=1}^{j+2} p_{i,j+1}\langle L_i\rangle\right)\nonumber\\
		&= 1 + p_{j+1,j} + \sum_{i=1}^{j} p_{ij}\langle L_i\rangle + p_{j+1,j}\sum_{i=1}^{j+2} p_{i,j+1}\langle L_i\rangle\nonumber\\
		&= 1 + p_{j+1,j} + \sum_{i=1}^{j} \left(p_{ij} + p_{j+1,j}p_{i,j+1}\right)\langle L_i\rangle + p_{j+1,j}\sum_{i=j+1}^{j+2} p_{i,j+1}\langle L_i\rangle\nonumber\\
		&= 1 + p_{1j}\langle L_1 \rangle + \sum_{i=2}^{h-1} p_{ij}\langle L_i \rangle + p_{hj}\langle L_h\rangle\nonumber\\
		&= 1 + p_{1j}\langle L_1 \rangle + \sum_{i=2}^{h-1} p_{ij}\langle L_i \rangle + p_{hj}\langle L_h\rangle\nonumber\\
\end{align}
Next we insert the damage roll probabilities. They depend on the max hit so once again we treat the two cases $h \leq m$ and $h>m$. Recall also that $L_1 = \frac{m+1}{am}$.\\
\textbf{Case} $h \leq m+1 \implies \Pr{X \geq h} = \frac{a}{m+1}(m-h+1)$.
\begin{align}
    \left(\frac{ma}{m+1} - \frac{\rho a}{m+1}\right)\L_h
	&= 1 + (1-\rho)\frac{a}{m+1}\L_1
	+ \sum_{i=2}^{h-1}\left((1-\rho)\frac{a}{m+1} + \rho\frac{a}{m+1}\right)\L_i\nonumber\\
    \implies (m - \rho)\L_h
	&= \frac{m+1}{a} + (1-\rho)\L_1 + \sum_{i=2}^{h-1} \L_i\nonumber
	= \frac{m+1}{a} - \rho\L_1 + \sum_{i=1}^{h-1} \L_i\nonumber\\
	&= \frac{m+1}{a} - \rho\frac{m+1}{ma} + \sum_{i=1}^{h-1} \L_i\nonumber
	= (m-\rho)\frac{m+1}{ma} + \sum_{i=1}^{h-1} \L_i\nonumber\\
	&= (m-\rho)\frac{m+1}{ma} + \sum_{i=1}^{\mathclap{(h-1)-1}} \L_i + \L_{h-1} \nonumber\\
    &= (m-\rho)\L_{h-1} + \L_{h-1} \nonumber\\
    \implies \L_h &= \frac{m-\rho+1}{m-\rho}\L_{h-1}
\end{align}
\textbf{Case} $h > m+1 \implies \Pr{X \geq h} = 0$
\begin{align}
    \left(\frac{ma}{m+1} - \frac{\rho a}{m+1}\right)\L_h
&= (1-\rho)\frac{a}{m+1}\sum_{i=h-m}^{h-1} (\L_i+1) + \rho\frac{a}{m+1}\sum_{i=h-m+1}^{h-1} (\L_i+1) + \left( 1 - \frac{ma}{m+1} + \frac{\rho a}{m+1}\right)\nonumber\\
    \implies (m-\rho)\L_h
    &= (1-\rho)\sum_{\mathclap{i=h-m}}^{h-1} (\L_i+1) + \rho\sum_{\mathclap{i=h-m+1}}^{h-1} (\L_i+1) + \frac{m+1}{a} - m + \rho\nonumber\\
    &= \sum_{\mathclap{i=h-m+1}}^{h-1} (\L_i+1) + (1-\rho)(\L_{h-m}+1) + \frac{m+1}{a} - m + \rho\nonumber\\
    &= \cancel{m} -\cancel{1} + \sum_{\mathclap{i=h-m+1}}^{h-1} \L_i + \L_{h-m} - \rho\L_{h-m} +\cancel{1} + \frac{m+1}{a} - \cancel{m}\nonumber\\
    &= m\L_1 - \rho\L_{h-m} + \sum_{\mathclap{i=h-m}}^{h-1} \L_i\nonumber\\
    &= m\L_1 - \rho\L_{h-m} + \sum_{\mathclap{i=(h-1)-m}}^{\mathclap{(h-1)-1}} \L_i - \rho\L_{h-m-1} + (\rho-1)\L_{h-m-1} + \L_{h-1}\nonumber\\
    &= (1 + m-\rho)\L_{h-1} - \rho\L_{h-m} + (\rho-1)\L_{h-m-1}\nonumber\\
    \implies \L_h &= \left(1 + \frac{1}{m-\rho}\right)\L_{h-1} - \left(\frac{\rho}{m-\rho}\L_{h-m} + \frac{1-\rho}{m-\rho}\L_{h-m-1}\right)\nonumber\\
\end{align}


