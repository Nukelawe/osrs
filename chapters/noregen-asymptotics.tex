Eq\ref{eq:explicitL} can be annoying to work with, especially for large $h$, so an approximation for the case $h \gg m$ is needed. Intuitively the length of a fight should have linear dependency on hitpoints as $h \rightarrow \infty$. This is because on average one can expect each hit to do $\frac{am}{2}$ damage and therefore it should take approximately $\frac{2h}{am}$ hits to kill an enemy with $h$ hitpoints. Due to overkill however, the last hit will tend to do slightly less than $\frac{am}{2}$ damage introducing a constant correction term to this linear approximation. We will now determine the magnitude of this constant by finding the asymptotic expansion for $\E{L_h}$.

To study the asymptotics we inspect the poles of the generating function which are simply the zeros of its denominator. We factorize out the $(z-1)$ terms in the denominator writing the generating function as
\begin{align}
	g(z)
		= \frac{(m+1)z}{a\big(m - (m+1)z + z^{m+1}\big)}
		= \frac{(m+1)z}{{a(z-1)}^2 Q(z)}
	\quad\mbox{where}\quad Q(z) = \sum_{k=0}^{m-1} (m-k) z^k\label{eq:ogf3}.
\end{align}
It is evident from eq\ref{eq:ogf3} that the generating function has a second order pole at $z=1$. The other poles $q_k$ are the zeros of the polynomial $Q(z)$ and cannot be found exactly. However, since $Q(z)$ is a polynomial with positive coefficients, we can use the Eneström-Kakeya theorem~\cite{kakeya} to obtain a lower bound
\begin{align}
	|q_k|
		\geq \max\limits_{0\leq k < m} \left(\frac{m-k}{m-(k+1)}\right)
		= \frac{m}{m-1} > 1\label{eq:kakeya}.
\end{align}
Now assume that for some $k$, $q_k$ is not a simple root and thus has a multiplicity greater than 1. Then also the derivative of the denominator of $g(z)$ must have a zero at $q_k$, that is
\begin{align*}
	(m+1)q_k^m - (m+1) = 0
	\implies q_k^m = 1
	\implies |q_k| = 1
\end{align*}
This contradicts the lower bound (eq\ref{eq:kakeya}) we just found implying that the poles $q_k$ are all simple.

Having discussed the properties of its poles we now express the generating function as a partial fraction
\begin{align}
	g(z) &=
		\frac{\alpha}{{(z-1)}^2} + \frac{\beta}{z-1}
		+ \sum_k \frac{\gamma_k}{q_k-z}
	\label{eq:ogf4}
\end{align}
where $\gamma_k$ are complex number constants and
\begin{align}
	\alpha
		&= \lim\limits_{z \rightarrow 1}{(z-1)}^2 g(z)
		 = \lim\limits_{z \rightarrow 1} \frac{(m+1)z}{aQ(z)}
		 = \frac{m+1}{aQ(1)},\label{eq:alphamid}\\
	\beta
		&= \lim\limits_{z \rightarrow 1}\frac{\d}{\d z}\Big({(z-1)}^2 g(z)\Big)
		 = \lim\limits_{z \rightarrow 1} \frac{m+1}{aQ(z)}\left(1 - \frac{zQ'(z)}{Q(z)}\right)
		 = \alpha\left(1 - \frac{Q'(1)}{Q(1)}\right)\label{eq:betamid}.
\end{align}
The values of the polynomial $Q(z)$ and its derivative $Q'(z)$ at $z=1$ evaluate to $Q(1) = \frac{1}{2}m(m+1)$ and $Q'(1) = \frac{1}{6}m(m+1)(m-1)$. Substituting them back into eq\ref{eq:alphamid} and eq\ref{eq:betamid} yields
\begin{align}
	\alpha = \frac{2}{am} \quad
	\beta = \frac{2(4-m)}{3am}.
\end{align}
Now we expand each term in eq\ref{eq:ogf4} as a power series around $z=0$.
\begin{align*}
	g(z)
		\nonumber&= \frac{2}{am}\frac{1}{{(1-z)}^2} + \frac{2(m-4)}{3am}\frac{1}{1-z} + \sum_k\frac{\gamma_k}{q_k}\frac{1}{1-\frac{z}{q_k}}\\
		\nonumber&= \frac{2}{am}\sum_{n=0}^\infty (n+1) z^n + \frac{2(m-4)}{3am}\sum_{n=0}^\infty z^n + \sum_k \frac{\gamma_k}{q_k}\sum_{n=0}^\infty \frac{z^n}{q_k^n}\\
		&= \sum_{n=0}^\infty \bigg(\frac{2}{am}\Big(n + \frac{m - 1}{3}\Big) + \sum_k \frac{\gamma_k}{q_k^{n+1}}\bigg) z^n
\end{align*}
Remembering that $\E{L_h}$ is the coefficient of the $h$th order term in the series expansion of the generating function we get
\begin{align}
	\E{L_h}\label{eq:asymptotic1}
		= \frac{2}{ma}\bigg(h + \frac{m-1}{3} + \sum_k \frac{\gamma_k}{q_k^{h+1}}\bigg).
\end{align}
which is the asymptotic expansion. By using the triangle inequality and eq\ref{eq:kakeya} the exponentially vanishing term can be bounded as follows.
\begin{align}
	\bigg|\sum_k \frac{\gamma_k}{q_k^{h+1}}\bigg|
	\leq \sum_k \frac{|\gamma_k|}{|q_k|^{h+1}}
	\leq \sum_k |\gamma_k|{\Big(\frac{m-1}{m}\Big)}^{h+1}
	= {\Big(\frac{m-1}{m}\Big)}^{h+1}\sum_k |\gamma_k|
\end{align}
Since $\sum_k|\gamma_k|$ is just a positive real number we see that $\sum_k \frac{\gamma_k}{q_k^{h+1}} \in O\Big({\big(\frac{m-1}{m}\big)}^h\Big)$. Dropping the exponentially vanishing term yields the asymptotic approximation.
\begin{align}
	\boxed{\E{L_h}
		\sim \frac{2}{ma}\Big(h + \frac{m-1}{3}\Big)
	}\label{eq:asymptoticAppr}
\end{align}
Figure~\ref{fig:apprComparison} illustrates the quality of the asymptotic approximation for various values of $h$ and $m$.
\begin{figure}[h]
    \includegraphics[scale=1.1]{plots/apprComparison.pdf}
	\caption{Value of the error term $\sum_k \gamma_k q_k^{-h-1}$ for various values of $m$ and $h$. Blue color indicates that the asymptotic approximation (eq\ref{eq:asymptoticAppr}) is \emph{smaller} and red that it is \emph{greater} than the true value of $\E{L_h}$. Clearly the best accuracy is obtained when $h \gg m$.}\label{fig:apprComparison}
\end{figure}
