Consider a sequence of $n$ fights and assume that throughout all of them the fight parameters $a$, $m$ and $h$ stay unchanged. If we denote the length of the $i$th fight in this sequence by $l_i$ then the time taken by all the fights together is $T_A(l_1+\cdots+\l_n)$. Likewise, if we denote the hitpoints regenerated during the $i$th fight by $r_i$ then the total damage dealt is $nh + r_1+\cdots+r_n$.
We define the \emph{kill rate} and the \emph{damage rate} as
\begin{align}
	v_k &= \lim\limits_{n\rightarrow\infty} \frac{n}{T_A(l_1 + l_2 + \cdots + l_n)}
		= \lim\limits_{n\rightarrow\infty} \frac{1}{T_A\overline{l_n}}\\
	v_d &= \lim\limits_{n\rightarrow\infty} \frac{h+r_1+\cdots+r_n}{T_A(l_1 + l_2 + \cdots + l_n)}
		= \lim\limits_{n\rightarrow\infty} \frac{h+\overline{r_n}}{T_A\overline{l_n}}
\end{align}
respectively, where $\overline{l_n} = \frac{1}{n}(l_1+\cdots+l_n)$ is the average number of hits to kill an enemy and $\overline{r_n} = \frac{1}{n}(r_1+\cdots+r_n)$ is the average hitpoints regenerated. Since both $l_i$ and $r_i$ are independent, identically distributed random variables we get by the law of large numbers that $\overline{l_n} \rightarrow \E{L_h}$ and $\overline{r_n} \rightarrow \E{R}$ as $n\rightarrow\infty$ and thus
\begin{align}
	v_k = \frac{1}{T_A\E{L_h}} \quad v_d = \frac{h + \E{R}}{T_A\E{L_h}}.\label{eq:dps}
\end{align}
When $T_A$ is in seconds, $v_d$ is the DPS (damage per second). The definition of the the rate quantities as long time limits matches with the common notion of DPS and thus makes them sensible measures of efficiency.
