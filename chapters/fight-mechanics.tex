\section{Fight mechanics}\label{chap:fightMechanics}
A fight consists of two parties, attacker and the enemy. We only consider the damage dealt by the attacker on the enemy and not vice versa. In our model the enemy is attacked periodically until its remaining hitpoints are 0. The attack rate is characterized by the \textit{attack period} $T_A$, which varies between weapons.
Every time the enemy is hit (once per time period $T_A$) the damage dealt is calulated as follows.
\begin{enumerate}
	\item The game determines if the hit will succeed by some random process which depends on various parameters, such as defensive bonuses of the enemy, offensive bonuses of the attacker and appropriate combat-related stats. We ignore the details of this process and just say the probability of the hit being successfull is $a$, which is called the \textit{accuracy}.
    \item If the hit was successful damage roll $M$, a uniformly distributed random number between 0 and $m$, is chosen, where $m$ is the \textit{maximum hit}. Like with accuracy we do not concern ourselves with the specifics of how the max hit is determined.
    \item If $M > H$, where $H$ is the remaining hitpoints of the enemy, the damage is capped to $H$ and the final damage dealt is $\min(M,H)$. This is the overkill effect.
\end{enumerate}

Unlike the damage calculation process, regeneration is not fundamentally random. It happens by periodically incrementing the remaining hitpoints by one until they are full. The period, $T_R$, of this healing cycle can vary between enemies giving rise to different regeneration rates. At the beginning of a fight the state of the healing cycle can be assumed to be unknown and thus treated as a random variable. This way only the timing of the first heal is random and the rest are perfectly periodic.

Throughout this study the parameters $a$, $m$, $T_A$ and $T_R$ will be treated as constants. The periods $T_A$ and $T_R$ can both be assumed integers if using gameticks as units.
