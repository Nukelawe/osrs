Figure~\ref{fig:mhValues} shows how $\E{L}$, $\E{R}$, $v_k$ and $v_d$ depend on the maximum hit $m$ and enemy hitpoints $h$. The $\E{L}$ plot illustrates the findings in Chapter~\ref{chap:noregenAsymptotics} that the fight's length increases linearly with $h$ and is inversely proportional to $m$. The number of hitpoints $\E{R}$ correlates very well with $\E{L}$, which was to be expected since $\E{R} = \rho(\E{L} + \E{N_{h,h}})$ and $\E{N_{h,h}}$, the number of times state $h$ is visited, is relatively small for slow regeneration rates. Likewise, the kill rate is simply the inverse of $\E{L}$, so its values are not particularly interesting.
Definitely the most interesting of the plots in Figure~\ref{fig:mhValues} is the one for the damage rate $v_d$, as it clearly displays the region in which the effects of overkill are the most prominent. At low hitpoints increasing the maximum hit barely affects the damage rate, but as the enemy hitpoints increase the effects of overkill vanish and the max hit dependency becomes linear.
\begin{figure}[h]
	\centering
    \includegraphics[scale=1]{plots/mhValues.pdf}
	\caption{The expected number of hits to kill $\E{L}$, number of hitpoints regenerated $\E{R}$, kill rate $v_k$ and damage rate $v_d$ for various values of max hit and enemy hitpoints. The rates are given using game ticks as units of time. The plots were made using the effective hitpoint approximation (eq\ref{eq:Leffhp}) with parameters $T_A=4$ (unarmed, scimitar, etc.), $T_R=100$ (most common regeneration period) and $a=0.5$.}\label{fig:mhValues}
\end{figure}

Figure~\ref{fig:arValues} shows how $\E{L}$, $\E{R}$, $v_k$ and $v_d$ depend on the relative regeneration rate. As we know from the analysis of the regenerationless case, the length of a fight depends on the accuracy as $\E{L} \propto a^{-1}$ and consequently $v_k,v_d \propto a$. Figure~\ref{fig:arValues} shows that around the critical point $\frac{2\rho}{am} = 1$ this relation breaks down. When the regeneration rate is larger than damage rate, $\frac{2\rho}{am} = 1$ and the accuracy dependency satisfies a power law $v_k \propto a^r$, where $r$ is some constant that depends on $m$ and $h$. A particularly violent increase in the fight's length is observed for the curve $m=3$, $h=50$, since now it takes multiple successful lucky hits in succession to lower the enemy hitpoints to 0 before regeneration brings them back up.
In the case $m=50$, $h=3$, we see a deflection \emph{downwards} in the slope of $\E{R}$. Around the critical point this corresponds to a fight in which the accuracy is terrible, but when a successful hit occurs, it inflicts a lot of damage killing the enemy with a high probability. As the accuracy decreases (towards right), such a fight will spend an increasingly large proportion of its duration in the fully regenerated state.
\begin{figure}[h]
	\centering
    \includegraphics[scale=1]{plots/arValues.pdf}
	\caption{$\E{L}$, $\E{R}$, $v_k$ and $v_d$ plotted against the relative regeneration rate $\frac{2\rho}{am}$. For each curve $T_A=4$, $T_R=100$, $m$ and $h$ were held constant while the accuracy $a$ was varied between 0 and 1 to obtain different relative regeneration rates. The curve cutoff point is where $a=1$.}\label{fig:arValues}
\end{figure}
