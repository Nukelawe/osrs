Equation~\ref{eq:noregenRecursion} is a linear recurrence of order $m$ with constant coefficients. It can be solved using a generating function $g(z)$, which when expanded as power series around $z=0$ has $\E{L_h}$ as its coefficients:
\begin{align}
	g(z) = \sum_{h=1}^{\infty} \E{L_h} z^h\label{eq:ogf0}.
\end{align}
The generating function can be found by substituting the recurrence relation in place of $\E{L_h}$ in equation~\ref{eq:ogf0}.
\begin{align}
	g(z)
		&= \sum_{h=1}^\infty\bigg(\frac{m+1}{am} + \frac{1}{m}\sum_{j=1}^m \E{L_{h-j}}\bigg)z^h\nonumber\\
		&= \frac{m+1}{am}\sum_{h=1}^\infty z^h + \frac{1}{m} \sum_{h=1}^\infty\sum_{j=1}^m \E{L_{h-j}} z^h \nonumber\\
		&= \frac{m+1}{am}\sum_{h=1}^\infty z^h + \frac{1}{m} \sum_{j=1}^m\sum_{h=1}^\infty \E{L_h} z^{h+j} \nonumber\\
		&= \frac{1}{m}\bigg(\frac{m+1}{a}\frac{z}{1-z} + \sum_{j=1}^m z^j g(z)\bigg)\nonumber\\
		&= \frac{1}{m}\bigg(\frac{(m+1)z}{a(1-z)} + \frac{z - z^{m+1}}{1-z} g(z)\bigg)\nonumber\\
	\implies m(1-z)g(z) &= \frac{(m+1)z}{a} + \big(z - z^{m+1}\big)g(z)\nonumber\\
	\implies \big(m - mz + z^{m+1} - z\big)g(z) &= \frac{(m+1)z}{a}\nonumber\\
	\implies g(z) &= \frac{(m+1)z}{a\big(m - (m+1)z + z^{m+1}\big)}\label{eq:ogf1}
\end{align}
To read off the explicit formula for $\E{L_h}$ the generating function must be expanded back into a power series. We begin by writing it in the form of a geometric sum.
\begin{align}
	g(z) &= \frac{\frac{m+1}{am}z}{1 - \frac{z}{m}(m+1-z^m)}\nonumber\\
	&= \frac{m+1}{am}z \sum_{n=0}^\infty {\Big(\frac{z}{m}\Big)}^n {\big(m+1-z^m\big)}^n\nonumber\\
	&= \frac{m+1}{am}z \sum_{n=0}^\infty {\Big(\frac{z}{m}\Big)}^n \sum_{k=0}^{n} {(m+1)}^{n-k}{(-1)}^k z^{mk} {n \choose k}\nonumber\\
	&= \frac{1}{a} \sum_{n=0}^\infty \sum_{k=0}^{n} {\Big(\frac{m+1}{m}\Big)}^{n+1} {\Big(\frac{-1}{m+1}\Big)}^k {n \choose k}z^{mk+n+1}\label{eq:ogf2}
\end{align}
$\E{L_h}$ is now the coefficient of the term $z^h$. Inspecting the exponent of $z$ in equation~\ref{eq:ogf2} reveals that for the coefficient of interest the indices $k$ and $n$ must satisfy $h=mk+n+1$. For each $k$ there is exactly one valid $n$, namely $n = h-mk-1$. Therefore, the coefficient of the $h$th order term in the series expansion of $g(z)$ is
\begin{align}
    \E{L_h} &= \frac{1}{a}\sum_{k \in \mathcal{I}} {\Big(\frac{m+1}{m}\Big)}^{h-mk} {\Big(\frac{-1}{m+1}\Big)}^k {h-mk-1 \choose k} \nonumber
\end{align}
where $\mathcal{I}$ is the set of indices $k$ such that
\begin{align*}
    0 \leq k \leq n = h-mk-1
    \iff (m+1)k \leq h-1
    \iff k \leq \frac{h-1}{m+1}.
\end{align*}
Since $k$ must also be an integer the index set becomes $\mathcal{I} = \left\{0,1,\ldots,\left\lfloor{\frac{h-1}{m+1}}\right\rfloor\right\}$ and the explicit solution to the recurrence can be written as
\begin{align}
	\boxed{\E{L}
		= \frac{1}{a}
		\sum_{k=0}^{\left\lfloor{\frac{h-1}{m+1}}\right\rfloor} {\left(\frac{m+1}{m}\right)}^{h-mk} {\left(\frac{-1}{m+1}\right)}^k {h-mk-1 \choose k}
	}\label{eq:explicitL}
\end{align}
where the index $h$ in $\E{L_h}$ has been omitted as the fight is assumed to start from full hitpoints.
For small $h$ equation~\ref{eq:explicitL} simplifies to the geometric sequence. In particular,
\begin{align}
	\E{L}
		= \frac{1}{a}{\left(\frac{m+1}{m}\right)}^h \quad \mbox{if }h \leq m+1.
	\label{eq:geomL}
\end{align}
Note how the accuracy dependency can easily be factored out from equation~\ref{eq:explicitL}. This gives a nice interpretation for $a\E{L}$ as the number of \emph{successful hits}, i.e.~hits that pass the accuracy check.
