We described the damage and natural regeneration mechanics in their full generality using an absorbing Markov chain. The quantities $\E{L}$ (number of hits to kill) and $\E{R}$ (number of hitpoints regenerated) were then shown to be solutions to linear systems of as many equations as the enemy has maximum hitpoints. Because solving such linear systems is computationally intensive for enemies with high hitpoints, we derived a variety of approximations, mainly focusing on small regeneration rates.

Treating in detail the special case of no regeneration we were able to get a good understanding of the effects of overkill. We derived an exact analytic solution (equation~\ref{eq:explicitL}) for the number of hits to kill the enemy and also found a convenient asymptotic approximation (equation~\ref{eq:asymptoticAppr}) for when enemy hitpoints is large compared to the attacker maximum hit. By applying the so called effective hitpoints argument, the regenerationless case was then extended to also approximately account for regeneration. For practically all realistic combat scenarios, the relative errors of the resulting effective hitpoints approximation were observed to be well under $1\,\%$, mostly under $0.1\,\%$. A more rigorous application of the effective hitpoints argument is still worth considering in the future as it might produce even better approximations.

We also briefly discussed a less restrictive assumption that still succeeds in triangularizing the transition matrix. However, because the resulting recurrence relations were more cumbersome to handle and the effective hitpoints argument was difficult to apply, we did not pay much attention to this approximation. Future work might focus on this or other ways of triangularizing the transition matrix. Some thought could also be given on generalizing the asymptotic approximation (equation~\ref{eq:asymptoticAppr}) to consider regeneration.

To conclude, we note that the treatment of the relative simple combat mechanics in Oldschool Runescape can be quite involved. There are also plenty of other combat mechanics that could be considered such as fights against multiple enemies at once, fights against enemies that attack back, poison damage and special attacks.

