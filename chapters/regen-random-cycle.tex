Instead of regenerating deterministically at realistic time intervals we could assume that the healing happens right after each hit with such a probability that the correct regeneration rate is achieved. This reduces the complexity of equation~\ref{eq:regenrecurrence} significantly as it removes the $\tau$-dependence entirely. We define the \emph{regeneration rate} as
\begin{align}\label{eq:regenProbability}
    \rho = \frac{T_A}{T_R}
\end{align}
and interpret it as the probability that a regeneration event occurs during an attack cycle.

In this model, there are two ways of lowering the hitpoints by $k$: hit $k$ and heal 0 or hit $k+1$ and heal 1. In terms of the regeneration probability $\rho$ and the damage roll $X$ the transition probabilities are
\begin{align}
    p_{ij}
         &= \begin{cases}
			 \Pr{X = j-i} + \rho \Pr{X = j-i+1} \quad &\mbox{if } i = h \\
            (1-\rho)\Pr{X = j-i} + \rho \Pr{X = j-i+1} \quad &\mbox{if } 1 < i < h \\
            (1-\rho)\Pr{X = j-i} \quad &\mbox{if } i = 1 \\
            \Pr{X \geq j-i} \quad &\mbox{if } i = 0
        \end{cases}\label{eq:damageDistributionRegen}.
\end{align}
Notice that transitioning to state 0 does not depend on regeneration because dead enemies cannot regenerate. For the same reason it is impossible to land in state 1 by first reaching 0 hitpoints and then healing. Transition probability for going to state $h$ on the other hand, is different because it is not possible to heal past maximum hitpoints.

Naturally it is still possible for the remaining hitpoints to climb up the state space making a recursive solution impossible. However, because of the eliminated dependence on the pahse of the regeneration cycle the walker is now fully described by equation~\ref{eq:fightLengthRecursion} just like in the regenerationless case. This reduces the size of the linear system down to $h$ equations, which is a much more manageable number for practical calculations.
